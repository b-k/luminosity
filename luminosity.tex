\documentclass{article}
\usepackage{natbib}
\begin{document}
\title{Illuminating poverty: Using sattelite light detection for a poverty indicator}
\maketitle
\begin{abstract}
Sattelite data includes areas that are difficult to reach via surveys or traditional big
data sources like cash registers and computers. We consider the use of luminosity data
from space as a correlate to poverty, using data from Bangladesh and Guatemala.
\end{abstract}

\section{Introduction}

\citet{chen:pnas} wrote on using luminosity as a measure of GDP.

\section{Model and methods}

\section{Results}
The raw correlation between luminosity and our poverty measure is pretty good

We ran a few regressions, including obvious correlates like percent with access to
electricity and population density, and found that poverty is still well-predicted by
log luminosity, and log luminosity is well-predicted by poverty.

Here are some extremely attractive maps to get the point across.

We then did the predictive exercise, using luminosity measures from 2001--2005 in a
training subset to predict poverty in 2005. We did OK.

\section{Conclusion}


\bibliographystyle{plainnat}
\bibliography{luminosity}

\end{document}
